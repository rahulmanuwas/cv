\cvsection{Projects}
\begin{cventries}	
	
  \cventry
  	{Self}
  	{Deep Q-network for Catch}
  	{Gurugram}
  	{May 1 - Present}
  	{
  		\begin{cvitems}
  			\item {Using an existing code to build a Deep Q-network that can play Catch using Pygame and Keras. The video can be found \href{https://goo.gl/qxjbgS}{\textbf{here}}. This utilizes a model-free reinforcement learning technique called Q-learning; which can be used to find an optimal action for any given state in a finite markov decision process.}
  		\end{cvitems}
  	}
  
  \cventry
	{\emph{GENPACT}}
	{Predicting category using NLP}
	{Gurugram}
	{Feb. 2017 - Mar. 2017}
	{
		\begin{cvitems}
			\item {Data coming from source has a category assigned by clients which was assigned incorrectly about 40\% of the time. Used regex, stemming and TF-IDF along with ANN and Logistic Regression to predict the correct category using text features from source. Achieved >80\% accuracy on a 4-class classification.}
		\end{cvitems}
	}

  \cventry
	{\emph{GENPACT}}
	{Calcuating Inshop-hours using Machine Learning}
	{Gurugram}
	{Jan. 2017 - Feb. 2017}
	{
		\begin{cvitems}
			\item {Applied multilinear regression techniques to calculate the number of Inshop-hours for resources to facilitate the supply chain business.}
	\end{cvitems} 
	}

%-----------------------------------------------------------------
  \cventry
    {University of Delhi}
    {Grid Computing and Performance Analysis}
    {New Delhi}
    {Aug. 2014 - Jan. 2015}
    {
      \begin{cvitems}
        \item {Designed and implemented a grid of computers connected on a network to match the computational power required to do scientific studies}
        \item {Measured computational performance of the grid via creating a python server and the file system was bench-marked using Hadoop.}
      \end{cvitems}
    }
    
  \cventry
    {Department of Physics \& Astrophysics, University of Delhi}
    {Probabilistic model for simulating CMS experiments}
    {New Delhi}
    {Aug. 2013 - Jan. 2014}
    {
      \begin{cvitems}
        \item {Simulated million diverse events that occur during the CMS experiment at the LHC by generating pseudo samples of data.}
        \item {Samples were obtained by inverse transform samplings (Box Muller \& Marsaglia). Various hypothesis were then tested over these data-sets to simulate detection of the Higgs.}
      \end{cvitems}
    } 
    
  \cventry
    {University of Delhi: Innovation Project}
    {Road Traffic Modeling and Simulation}
    {New Delhi}
    {Jul. 2012 - May 2013}
    {
      \begin{cvitems}
        \item {Developed a mathematical model to simulate real time traffic on selected roads of Delhi.}
        \item {Traffic Jams were analyzed to create a realistic simulation.}
        \item {Used recordings and simulations to obtain solutions for the Traffic. Exhaustive testing of new layouts and occasional modifications were made to the vehicle rules for the same region of space to obtain desired results.}
      \end{cvitems}
    }
    
  \cventry
    {Cluster Innovation Centre}
    {City Planning and Route Optimization}
    {New Delhi}
    {Aug. 2013 - Dec. 2013}
    {
      \begin{cvitems}
        \item {Developed a plan for a city by optimizing the available free space and the traveling routes using Graph Theory.} 		
        \item {The efficiency of Floyd-Warshall, Dijkstra and Genetic Algorithms were compared and a new shortest path algorithm was also proposed.}
      \end{cvitems}
    }
    
  \cventry
    {Cluster Innovation Centre}
    {Radial breathing mode analysis of SWCNT}
    {New Delhi}
    {Aug. 2012 - Dec. 2012}
    {
      \begin{cvitems}
        \item {Studied the normal modes of vibrations a Carbon Nanotube using Singular Value Decomposition(SVD) analysis on molecular dynamics data.}
      \end{cvitems}
    } 
    
  \cventry
    {Cluster Innovation Centre}
    {Development of GPS based SOS Tracking System}
    {New Delhi}
    {Aug. 2013 - Jan. 2014}
    {
      \begin{cvitems}
        \item {The GPS module SIM908 interfaced on Arduino platform is used to trigger an SOS signal depending upon the pulse rate of a person at the time of accident like scenario. This device module can prove extremely useful especially when help is needed urgently and the victim is unconscious.}
      \end{cvitems}
    }   
    
\end{cventries}
